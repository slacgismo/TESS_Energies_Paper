\section{Introduction}

% Motivation and contributions of this paper
Residential distribution systems are a potential source of system flexibility, particularly in demand. This potential can be realized by an approach called transactive systems (TS), which  coordinates residential distributed energy resources (DER) through a market-based mechanism. 
In this work, we present TESS, a modularized platform for the implementation of TS, which enables the deployment of adjusted market mechanisms, economic bidding, and the potential entry of third parties. TESS thereby opens up current integrated closed-system TS, allows for the better adaptation of TS to power systems with high shares of renewable energies, and lays the foundations for a smart grid with a variety of stakeholders.
Furthermore, we describe TESS as we have modified it for a field implementation within the service territory of Holy Cross Energy in Colorado. Importantly, our specification addresses challenges of implementing TS in existing electric retail systems, for instance, the design of bidding strategies when a (non-transactive) tariff system is already in place.

% Definition and brief history of TS
Transactive energy systems generally use market processes and automated device bidding to coordinate supply and demand and provide devices with control signals using price-based dispatch. The ground-breaking work in the GridWise Olympic Peninsula project demonstrated in the field that flexible devices can be controlled using a simple price-discovery mechanism to reduce energy procurement cost and manage congestion constraints to the distribution grid (see \citet{PNNL2006} and \citet{hammerstrom_2008}). Heating, Ventilation, and Air Conditioning (HVAC) systems and water heaters submitted bids for dispatch to a market operator. These bids were based on heuristic bidding functions, which resulted in higher bids for customers with higher priority.
The demand bids were settled with the available supply and resulted in a locational price for energy that equated marginal benefit and marginal cost. 
This price generally equalled the marginal cost of electricity procurement if import of electricity was unconstrained, and deviated from it when transmission capacity was scarce. Through this mechanism, the operators of the TS were able to reduce energy procurement cost and manage limited import capacity from the bulk power system.

% More recent work
More recent work on TS has extended this work in the following economic aspects. 
One line of work concerns device bidding.
\citet{PNNL2006} and \citet{hammerstrom_2008} initially implemented a heuristic bidding function for HVAC systems. This approach was, for instance, extended by \citet{adhikari_simulation_2016} who analyzed different approaches, with and without pre-cooling.
However, neither approach considered customers' incentives explicitly, e.g. with regard to comfort.
Instead, \citet{Arlt2020} has proposed an economically motivated bidding function.
\citet{widergren_transactive_2017} further addressed this open question by describing a general framework of economic, incentive-based bidding of demand.
\citet{lian_transactive_2020} have additionally considered strategic bidding of devices, for instance under a Stackelberg model.
A second line of contributions concerns the inclusion of additional device types. While the initial pilots included HVACs and water heaters on the demand side, more recent work also included other DER.
Electric vehicles were considered by \citet{behboodi_integration_2016} and \citet{behboodi_electric_2016}. 
Photovoltaics (PV) was analyzed in \citet{sajjadi_transactive_2016}, \citet{ableitner_user_2020}, and \citet{mengelkamp_decentralizing_2018}, for instance.
\citet{parandehgheibi_two-layer_2017} proposed a transactive controller for electric storage.
Third, several contributions have discussed different market mechanisms other than the central double auction. \citet{fuller_analysis_2011} and \citet{Arlt2020} have analyzed the performance of double auctions in detail. \citet{widergren_residential_2014} have further worked with multiple feeders and transactive nodes, resulting in locational prices. Another extension has been opening up the TS architecture for peer-to-peer trading, e.g. as summarized by \citet{zhang_review_2017} and \citet{sousa_peer--peer_2019}.
Institutional concerns regarding peer-to-peer trading have, for instance, been addressed by \citet{masiello_sharing_2016}.
%Valuation: PNNL Part II \citet{lian_part_2017}, \citet{lian_transactive_2018} formulate a general valuation framework of economic bidding but do not specify their framework for any device or DER.
\citet{abrishambaf_towards_2019} provides a taxonomic survey of ongoing research efforts with regard to TS and lists relevant pilot projects in the US and in Europe. 

% Gap for TESS as a platform
The summarized literature shows that TS have been adapted continuously to reflect the new economic opportunities and challenges posed by power systems with increasing shares of renewable energy and DER.
However, we still see the need to progress on economically-founded bidding functions and new market designs that are better able to address the new challenges of renewable energy systems with high volatility and uncertainty of generation.
In this work, we therefore introduce the Transactive Energy Service System (TESS) platform, which modularizes TS architectures in field deployment, and present its economic building blocks. 
The platform allows for continuous development of the essential components of TS.
Furthermore, the TESS design concept breaks up the closed system design that TS have usually followed in field deployment. In particular, TS have been designed as a system for the utility to manage procurement cost and dispatch devices according to their priority. However, future energy systems are likely to be shaped by a variety of stakeholders offering innovative business models, including manufacturers of smart home systems and flexible devices or load aggregators. This evolution also introduces the possibility for customers to deviate from the dispatch determined by the TS instead of allowing for direct price-based control through the utility. 
Future TS must take these potential developments into account and be open for interactions that have yet to be defined. With TESS, we hope to provide a starting point for this development.

% Brownfield
Moreover, we observe that most of the implemented TS operate on the assumption that the TS fully substitutes the existing tariff system (``greenfield''). 
In that case, devices bid their marginal value and supply is offered at marginal cost, typically the real-time wholesale market price (e.g., \citet{PNNL2006}, \citet{katipamula_transactive_2017}). This context, however, does not reflect the incentive structure customers are facing in real-world implementations, with existing tariff schemes and outside options affecting the opportunity costs facing an individual customer.  
Past deployments therefore needed to operate under experimental conditions: customers were either assigned to an experimental tariff or received side-payments for a separate billing through the TS. 

The presence of fixed retail tariffs as well as net-metering (``brownfield''), however, change the design requirements and conditions for TS operations. They also change customers' incentives for participation. 
Although a TS provides an alternative compensation mechanisms for DER that could, for example, replace net metering, existing tariff schemes or regulations are hard to change and often require approval from political bodies or regulatory commissions. 
Moreover, while some of these tariff components divert customers from efficient economic dispatch, they can serve other socioeconomic objectives or are meant to push technology adoption. It is yet unclear how such objectives could be pursued under a TS.
Finally, power systems -- with their technical, economic, and social aspects -- are complex systems. Many of the agents involved, whether customers, utilities, or regulators, are risk-averse. 
For these reasons, the requirements of a greenfield TS has potentially hindered broad acceptance and deployment of TS in utilities across the US and explains why -- while experimental field deployments have been considered successful -- TS have not been broadly adopted in the field.
Therefore, systematic approaches for brownfield deployment might be necessary to establish the technology, learn about operational challenges, and provide the road to a true ``smart grid''.

In a field implementation of TESS within the service territory of Holy Cross Energy (HCE) in Colorado, we have been confronted with such brownfield conditions.
In this paper, we therefore describe TESS as we have modified it for HCE. We focus on photovoltaics (PV) systems which are a widely adapted DER and elaborate how we designed bidding functions for PV given that customers usually benefit from net-metering. Under net-metering, residential customers only pay for their net consumption, i.e. the value of solar generation equals the fixed retail rate, from the perspective of the customer.
This establishes an important baseline for customers' bidding behavior who aim to be better off by participating in a TS or otherwise leave it.
We argue that, for a TS to be beneficial to all parties, these economic incentives play a pivotal role. This claim is particularly true in brownfield deployments, where customers are already part of an existing tariff system and have outside options, such as not participating in the TS.
%\citet{sajjadi_transactive_2016} base their bid on the average cost of a kWh generated, including total cost of installation. We do not follow their approach as this number does not capture the effects on individual incentives of the marginal benefit of participating in the TS, nor does it capture additional existing tariff conditions as outside options. In a European context, \citet{ableitner_user_2020} implemented a local electricity market. While owners of solar panels could sell at a guaranteed feed-in-tariff, they find that some would be willing to sell at lower costs. \citet{mengelkamp_decentralizing_2018} automate customers to bid marginal supply costs between the feed-in-tariff and the retail rate.
%In the context of fixed feed-in tariffs, comparable considerations have been made by \citet{mengelkamp_decentralizing_2018} who automate customers to bid marginal supply costs between the feed-in-tariff and the retail rate. 

% Combine both contributions
In combination, by opening up TS and enabling economic bidding by considering available outside options, we facilitate mutually-beneficial exchange of electricity and flexibility in the TESS design by focusing on actual \emph{price discovery} instead of generating a control signal specific to a closed system. Price discovery is the process by which producers and consumers interact to reveal not just the accounting cost of operating a supply resource, but also the actual marginal value of consumption, the opportunity costs facing both buyers and sellers, and the value of the alternatives available to them. %{\color{red}ALSO OUTSIDE OPTIONS} 
%The goal of the TESS project is to design, develop, test, and validate retail-level Transactive Energy systems that are dominated by behind-the-meter renewable energy resources and energy storage resources. Our definition differs from the cost-based ``prices to devices'' concept by including demand-side and DER bidding based on willingness to pay -- eliciting ``prices from devices'' to reflect demand-side value more directly in the market-clearing prices used for dispatch.

We proceed as follows: In \cref{sec:background}, we summarize the history of transactive systems and the economic theory underlying our TS design. In \cref{sec:standard_design}, we describe the TESS standard design. Then, in \cref{sec:hce}, we give a description of Holy Cross Energy and the conditions of the TESS field experiment. We follow up with the modifications to our transactive system and discuss the challenges for implementation in \cref{sec:challenges}. We conclude in \cref{sec:conclusion}.

