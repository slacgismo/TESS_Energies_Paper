\section{Background}\label{sec:background}

The TESS platform builds on previous TS work. In this section we summarize that history (Section \ref{sec:tehistory}) and the theoretical context that is the economic foundation of our design (Section \ref{sec:teecon}).
This background provides the foundation for Section \ref{sec:standard_design} to show how TESS opens up the closed control architecture of current TS, as well as how it embodies economic theory in its emphasis on price discovery and price-based dispatch to reflect demand and DER value, opportunity costs, and outside options more explicitly in electric systems.

\subsection{Transactive Energy Background}\label{sec:tehistory} 

Transactive energy was originally conceived of as \emph{transactive control} to address the problem of integrating large numbers of small resources optimally into electric power system operations (TE 1.0).
TE 1.0 implemented the ``prices to devices'' concept by pushing a cost-based price signal to devices, observing the response, and changing the price signal iteratively to bring quantity supplied and quantity demanded into equilibrium (a process economists will recognize as Walrasian \emph{tat\^{o}nnement}). This utility-generated price signal tends to be cost-based and supply-focused, reflecting demand only to the extent that devices respond iteratively, with an objective function of achieving a particular system control objective.

Such iterative price discovery methods often introduce feedback delays that can lead to price and control instability. Hence they are not well-suited to situations where fast-acting control of DERs is desired.
Eliminating the iterative process requires large amounts of potentially private information about those resources' supply and demand functions, such as would be required for a formal numerical optimization solution. This problem has become even more prominent as utilities struggled to integrate a growing variety of behind-the-meter demand response and DERs, including photovoltaics, electrified end-uses that were previously fossil-based, electric vehicles and their chargers, and energy storage into their system operations using conventional utility programs such as energy efficiency and demand-side-management. 

The transactive energy market design developed for the GridWise Olympic Peninsula Testbed Demonstration Project \citep{hammerstrom_2008} moved beyond TE 1.0 (TE 2.0). One dimension of the project tested a real-time market, implemented as a double clock auction buyers and sellers submit bids and offers simultaneously in a specified market time interval. The market clearing provided a price signal for automated device demand response and dispatch.
Automating device participation in such markets reduces transaction costs, elicits information about relative value and relative cost from agents around the edge of the distribution network via their devices, increases the frequency of participant activity, improves responsiveness, and makes autonomous price-based dispatch feasible. 

Despite the resemblance to a market mechanism, this setup differed from an economic mechanism in several respects. First, the bids did not reflect the actual economic value customers attributed to dispatch, but rather indicated a heuristic technical prioritization. The following equation represents the bidding function of an HVAC system, the main flexible device of the Olympic Peninsular experiment:

\begin{equation}
    b_t &= p_{mean} + (T_{current} - T_{set}) \frac{k p_{var}}{|T_{lim} - T_{set}|}
\end{equation}

The bidding functions converted comfort objectives, e.g., indoor air temperature $T_{current}$, to reservation prices $b_t$ based on the expected price mean and variance. As comfort decreased, the bid price rose above the mean price $p_{mean}$, and did so more if the price variance $p_{var}$ was greater. When comfort increased the bid dropped below the mean price, and did so more if the price variance was greater. The price mean and variance were taken over the past 24 hours, which was very effective at capturing the diurnal interaction of comfort demand and price fluctuations.
Bidder priority or indication of comfort could be set through $k$. $T_{lim}$ represents the minimum or maximum price and $ T_{set}$ the respective temperature setpoint, depending on the mode of the HVAC system (heating or cooling). 
The resulting bidding function is visualized in  Figure~\ref{fig:Transactive_DA} (right).
The aggregation of individual bids gives the demand function of the system (left).

\begin{figure}[t]
\centering
\includegraphics[scale=0.5]{images/TE_DA.png}
\caption{Transactive Double Auction (left) and Thermostat Bidding Strategy (right)}
\label{fig:Transactive_DA}
\end{figure}

However, the economic utility framework of the customers remained undefined. This heuristic approach usually creates circumstances in which devices place bids that do not reflect customers' actual marginal values and, as a result, customers might wish to deviate from a cleared bid. However, in the Olympic Peninsula project, customers could not change the TS-determined dispatch because devices were part of a closed technical control system. The only option left to customers was to override preferences (e.g., temperature setpoints) and essentially exit the TS as a flexible device for the duration of the override. In bidding functions, however, the design task is to express customers' marginal value of dispatch. Only when the bidding function reflects that marginal value do customers have no incentive to deviate from their allocation to dispatch. As a result, market clearing coincides with social welfare optimization. 

Second, because the TS was layered on top of their existing utility contract and was separate from it, customers made decisions as if no other tariff considerations interfered with their market participation (``greenfield''). Customer bidding behavior was therefore guided by the marginal cost of consumption to them (and, likewise, the marginal cost of supply for the utility). This approach ignores the complexity of outside options customers are facing, for instance because of an existing tariff system. Under a true market mechanism, customers would always consider all outside options and opportunity costs of participating in the TS.

TE 2.0, with a double auction and automated price-based dispatch, changes the paradigm to ``prices from devices'' to establish an informative market-clearing price that is then used for prioritization and dispatch. The TE 2.0 paradigm is more adaptable to unknown and changing conditions and better enables flexibility in the face of uncertainty at varying scales and time frames. It is thus well-suited to the operational challenges associated with increasing DER adoption.

\subsection{Transactive Energy Economic Theory}\label{sec:teecon}

TESS, our implementation of TE 2.0, employs a synthesis of economic market design and control systems, with an emphasis on demand-side and DER owner/operator bidding \citep{chassin2017thesis}. 
As in previous TS, it bases coordination on preference-based bids from both demand and supply agents. 
One of the most important and challenging dimensions of TS design is how the preferences of the demand-side and DER can be expressed and specified for automated bidding. 
\citet[Chapter IV]{Arlt2020} has shown how such a bidding function can be specified for HVAC systems.
Other TS models tend to fall in one of two categories: basing the device bidding on a cost-based objective, or basing the device bidding function on a technical heuristic or on an aggregate or market demand approximation \citep[Chapter IV]{Arlt2020}. Instead, TESS will enable the implementation of bidding functions that enable users to communicate their preferences for dispatch and their availability for flexible dispatch more directly.

The economic foundation of demand-side and DER bidding draws from multiple fields of economic theory \citep{kiesling_2021}. An essential feature of the economics of transactive energy is that people are distinct, with individual and private characteristics and subjective preferences over the goods and services they consume. Consumption bids should reflect their marginal willingness to pay. Producers, too, face opportunity costs that might be unknown and heterogeneous, and their perception of opportunity costs can even be subjective. Subjectivism in economic theory reflects the key insight that personal preferences and opportunity costs are private knowledge.

The preferences and opportunity costs of consumers and producers are also not knowable by or accessible to others, so coordination requires some mechanism that gives people incentives to communicate some of that subjective, private knowledge and turn it into information. That mechanism is the price system. A price system operating within a clear set of rules provides a decentralized mechanism for gathering information and learning about these preferences and opportunity costs, even among strangers (or their devices). The market process of mutual learning and decision-making enables prices to emerge that coordinate the actions and plans of all users of the system, incorporating both supply and demand information into that emergent price.

All price systems operate in an institutional framework, a set of rules, and the rules shape the incentives of participants. Effective price systems and market institutions can be challenging, or can fail to exist, due to transaction costs. When transaction costs are high and market transactions are costly to employ for coordination, other methods of organizing economic activity emerge, such as vertical integration. Technological change, most recently digitization and automation, have reduced transaction costs of using the price system relative to these other institutions \citep{kiesling_2016}. The TESS design takes advantage of these transaction cost reductions.

One institutional form for price discovery is auctions. Auctions are useful aspects of market design when the allocation problem is characterized by asymmetric information. In the case of electricity market design, the object for sale has different, subjective value to each potential bidder, a case known as a private value auction model. The double auction (DA), as discussed in Section \ref{sec:tehistory}, treats buyers and sellers symmetrically, enabling them to make simultaneous bids and offers. In a static DA, buyers submit bid schedules and sellers submit offer schedules of price and quantity \citep{friedman1993double}. The centralized auction platform performs a clearinghouse function, arranging all of the buyers’ schedules into a market demand curve and all of the sellers’ schedules into a market supply curve. 
This process enables a market price to emerge, determining quantity supplied and quantity demanded and clearing the market. 
Theoretical and experimental analysis suggests that the DA is a very efficient market design, enabling price discovery even in the presence of fragmented, private knowledge, where each participant knows only her own value/opportunity cost \citep{easley1993theories}.

Another relevant field of economics is mechanism design, a game-theoretic approach to examining situations in which individuals make decisions and interact when they have private knowledge that could affect the decisions they make and the ultimate outcomes. Mechanism design focuses on eliciting truthful revelation of some information about that private knowledge/type. Depending on the environment and the objective function of the principal, different mechanisms can perform better or worse at truthful revelation.
The general structure of mechanism design is to maximize some objective function across all $n$ individuals subject to two constraints: incentive compatibility (IC) and individual rationality or participation (IR), where $v_i$ is the value function for agent $i$, $\theta_i$ is agent $i$'s type, $g_i$ is agent $i$'s outcome, and $t_i$ is a monetary transfer to/from agent $i$.

\begin{equation}
IC: 	v_i(g^*_i(\theta_i,t_i(\hteta_i)) \geq v_i(g_i(\theta_i,t_i(\theta_i)) \qquad \forall i,g
\end{equation}

\begin{equation}
IR:	    v_i(g^*_i(\theta_i,t_i(\theta_i)) \geq 0
\end{equation}

The objective function varies depending on the specific context, but represents profit, surplus, or welfare maximization. The IC constraint ensures that each agent is better off truthfully revealing type in the outcome from the chosen mechanism compared to other alternative outcomes. The IR constraint ensures that each agent is better off participating in the system than not participating. 
Practical mechanisms that can be implemented are hard to identify, and testing such mechanisms is an essential step in overall market design. Experimental economics provides a framework for performing such market design tests, and field deployments are an example of these important tests.

These economic fields and principles inform our overall approach to market design for price discovery, and our design of device bidding functions taking into account opportunity costs and outside options to enable incentive compatible voluntary participation of agents in TESS.

