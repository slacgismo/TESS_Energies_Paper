\section{TESS Standard Design}\label{sec:standard_design}

In this section, we discuss which design requirements future TS need to address to integrate high levels of DER and prepare the path to a true, market-based smart grid.
We are designing the Transactive Energy Service System (TESS) platform to enable the implementation of these functionalities.
Specifically, they include all those found in the original Olympic demonstration, as well as (1) a standardized platform that reveals the full extent of demand-side and DER flexibility, (2) economically founded bidding functions, (3) a real-time market operation capable of operating concurrently at multiple time-scales, and (4) a flexible settlement mechanism. 
%The information systems support a wider range of capabilities such that any  bidding function is permitted, instead of the ``approved'' functions used in the Olympic and subsequent transactive system demonstrations. Accommodating heterogeneous bidding functions reflects the subjective nature of preferences and opportunity costs and enables a more decentralized platform that gives utilities, device manufacturers, and consumers more flexibility and autonomy than previously possible in transactive systems.
The following sections will detail each of them.
%Section \ref{sec:technical_setup} describes the technical setup of the TESS platform, Section \ref{sec:price_discovery} defines how the TESS design enables price discovery, and Section \ref{sec:time_scales} delineates the challenges of markets at different time scales and how the TESS design addresses those challenges. Section \ref{sec:devices} highlights a distinctive feature of the TESS design, the variety of devices that can participate and their bidding functions. Section \ref{sec:settlement} describes the TESS approach to financial settlement of transactions.

As a platform for implementing TS, the TESS modules described here are only part of a whole-of-operation approach which addresses the full life cycle of a transaction deployment with the objective to integrate distributed energy resources into the utility operation.  
More broadly, TESS will also include program development tools such as screening, financing, and engineering software, deployment tools such as participant identification, recruitment, enrollment, and bootstrapping products, as well as system operations tools such as wholesale, distribution, and settlement products to give the utility all the resources needed to successfully implement a TS.
%The standard TESS design further takes a modular approach that addresses the full life cycle of a transaction deployment, starting from program screening, financing and engineering design, through participant recruiting, system provisioning, hardware and software commissioning, all the way through to retail, wholesale, and settlement operations.
This approach will allow many different kinds of utility operations to be supported using the same system infrastructure, significantly reducing the barriers to adoption and the cost of deployment of TS. 

\subsection{Technical Setup of Market Operations}\label{sec:technical_setup}

The code and documentation of TESS are available at \url{https://github.com/slacgismo/TESS}. The current setup reflects the specification of TESS for our field deployment in Colorado, as described in the following \cref{sec:hce}.

Each customer has a home hub that manages the settings and actions of its devices on the customer's behalf. At each market interval, the relevant physical state information is measured by the meter or other relevant sensors and written to the TESS database on the cloud.
Based on this information, the agents representing the home hubs on the cloud update their physical states and form dispatch bids, i.e., how much they are willing to pay for consumption or the minimum price for supply. 
The agent representing the retailer submits wholesale market supply, with the price bid corresponding to the cost of supply and the quantity bid to the import capacity. This coordination with wholesale operations is essential to ensuring that the utility is maximizing the utilization of the wholesale agreements it has in place.
Shortly before the beginning of the market interval, the market operator agent collects all bids and determines the market result by constructing market demand and supply for that interval. 

The market is cleared in a welfare-maximizing way, where quantity supplied equals quantity demanded. The result is characterized by an equilibrium price and the equilibrium quantity.
After market clearing, home hubs compare their devices' bids to the equilibrium price. If the bid price is equal to or exceeds the equilibrium price (for demand) or is less (for supply), the bid was cleared. 

All agents are implemented on the cloud and values (measurements, bids, market results) are stored in the TESS database.

Periodically the utility must settle the accumulated account balances with all the participants in the TS, see \cref{sec:settlement}.

\subsection{TESS Design: Price Discovery Mechanisms}\label{sec:price_discovery}

%Price discovery occurs through the interaction of demand bids and supply offers on the market platform. 
The double clock auction mechanism is a key feature of standard TS design. Most TS are currently designed to discover energy prices per MWh, i.e., bid quantities in a given time period are measured in MW.  The analogy to wholesale energy markets is obvious, insofar as TS are often designed to mitigate distribution constraints.
The double auction used on most existing transactive projects, however, has important limitations that TESS seeks to overcome in its design. 

First, double auctions are unable to optimally clear resources that can act significantly faster than the auction interval, or are less flexible. The typical market interval is 5 minutes which is also implemented in TESS as a default. However, TESS allows an operator-specified market clearing interval.

Second, it is also difficult, if not impossible, to reveal any forward knowledge through the real-time double auction  mechanism, which precludes agents from making mutually beneficial forward commitments. To address these problems, the TESS design includes an order book mechanism that accepts two types of bids: market orders and limit orders. A market order allows a resource to dispatch immediately at the current price, whatever it is. A limit order allows a resource to be reveal its dispatch capabilities for any future time interval, contingent on a suitable ask or offer price.

Third, as more DERs enter the resource mix, particularly zero marginal cost resources, changing market design away from energy as the traded good may be useful. Three different resource attributes and their prices are mathematically related: energy prices (\$/MWh), storage prices (\$/MWh$^2$), and ramping prices (\$/MW) \citep{chassin2017thesis}. TESS' modular structure is designed to support these three distinct price discovery mechanisms simultaneously. 
%At the time of this writing, these are beyond scope of the standard transactive design and are topics for further research.

%The physical scaling of a transactive system can also be a difficult design and implementation problem for utilities, although usually it is not a problem. For example, in the Columbus demonstration project \citep{Widergren2014} the standard transactive system was deployed on 4 feeders without significant changes to the implementation.  However, the Pacific Northwest project \citep{hammerstrom2015pacific} encountered a number of important market design and operation issues that emerged due the need to coordinate the dispatch of retail resources across 11 utilities in a region without restructured electricity markets.
%
%TESS continues to focus on retail-level pricing mechanisms, but does not focus exclusively on using energy prices (\$/MWh) to manage distribution system constraints.  As mentioned in \ref{sec:teecon}, TESS is designed to permit discovery of other prices, such as storage prices (\$/MW$^2$) or ramping prices (\$/MW) that can be used to provide signals that can incentivize changes to the state of charge objective of a battery or the power setting of an inverter, respectively.  However, these prices signals have yet to be studied in detail in simulations or demonstrated in the field, and will be subject of future projects.

Ultimately, TESS will also provide multi-level market integration, from retail to feeder to wholesale.

\subsection{TESS Design: Participating Devices and Bidding Functions}\label{sec:devices}

In the standard design piloted by the Olympic project, thermostat heating/cooling devices, diesel generators, micro-turbines, white goods appliances, and municipal water pumping systems were the principal resources that could place bids and offers in the retail market.  At the time, these were the typical devices utilities hoped to dispatch using a retail real-time energy price. 
TESS enables a wide variety of resources to participate in the market, with custom bidding functions that can communicate both the technical capabilities of the resource and the preferences and opportunity costs of the owner.
More recently, new types of flexible DERs figure prominently in utility distribution system operations and TESS is therefore also supporting electric vehicle chargers and electric storage.

The bidding strategies devices used in the Olympic Peninsula project were selected by the system engineers, and not by the vendors or users of the devices. This design choice was necessary given the lack of experience with TS, if only to ensure that the system functioned properly. 
%For example, in the Olympic project, diesel generators had a very complex bidding strategy because of the 100 hour annual maximum run-time license.  
This means that the bids submitted potentially do not reflect the actual marginal value of dispatch which raises potential issues with incentive compatibility if consumers are given the capability of deviating from their bid after the market has been cleared \citep{li2020transactive}. 
Furthermore, high degrees of automation not only ensured strong demand response resource participation, but also protected against strategic bidding practices that could undermine the design objectives of the mechanism.
Nevertheless, the system designers recognized that ultimately the bidding strategies are the most important part of the system and the design, and selection of these strategies was a task that ought to be left to the device manufacturers and consumers in a decentralized system.

The TESS design brings this notion to full realization by allowing devices (through device manufacturers and third-party agent developers) to implement their own bidding strategies, and giving consumers more direct control over the settings that govern those strategies. For example, consumers can control their battery storage resource strategy by changing the preferred energy reserve (given their preference for resilience) and its sensitivity to price fluctuations (given their risk aversion).  Similarly, electric vehicles can be configured to charge more aggressively, ignoring prices fluctuations, or more economically by charging only when the price is low (\cite{behboodi2016electric}). 
%In addition, TESS differs from the transactive system demonstrated earlier because it does not standard the bidding strategies employed by difference classes of devices.  Instead, device manufacturers and third-party agent developers are free to offer any number of bid strategies from which the consumer can choose.  
Bid function heterogeneity further is an important and distinctive feature of TESS that improve price stability, and is likely to make it more difficult for one type of device or vendor to exercise any kind of market power.
The TESS design also anticipates the emergence of AI bidding agents, where learning and forecasting functionalities are included. This development is expected to bring additional diversity in the bidding strategies and add to the overall resilience and stability of the TS by avoiding many adverse ``flocking'' behaviors and resulting price volatility observed in the previously demonstrated TS.

Finally, if the market design is adjusted as described in \cref{sec:price_discovery}, bidding strategies need to be adjusted to incorporate the new set of rules.
In the case of order books, there is also a time dimension that must be included, and absent a price, the bid is considered simply a market order and the price of the next available resource is taken.

\subsection{TESS Design: Settlement Mechanisms}\label{sec:settlement}

While demand and supply coordinate in real time through a TS, the accounting to reconcile financial flows with transactions often takes place later (e.g., monthly). Several financial settlement systems have been demonstrated in previous TS, including a side-payment mechanism \citep{hammerstrom_2008} and a regulator-approved tariff \citep{Widergren2014}. 

TESS completes the settlement process in two steps. First, all the unsettled transactions are summed up for each participant. The utility then computes the total costs and revenues of TESS operations, to determine the net benefit of operating TESS during the settlement period.  The suitably chosen portion of total net benefit, e.g., 50\%, is then divided among all the participant \textit{pro rata} their net transactions for the period.  For example, if Alice earned \$100 during the period, and Betty earned \$50, then if the utility saw a net benefit of \$30, the payment to Alice would be \$110, and the payment to Betty would be \$55, and keep the remaining \$15. Note that this net benefit may be negative number, in which case it would reduce the participant payments in a similar manner. In addition, some participants may have negative payments, i.e., costs, in which case the credit may be offset those debits. 

In addition, TESS proposes a token-based settlement systems which also permits the use of new digital or crypto-currencies, if desired.
For instance, a certain number of tokens could be issued each month by the utility based on the expected value of the total welfare surplus from operations. Customers then participate in the transactive system using these tokens, so their token balances reflect their decisions and actions. At the end of the month the total surplus is calculated and used to assign a value to the tokens. Each participant is then compensated according to how many tokens they hold and their individual bill credits awarded accordingly. This approach uses cooperative game theory concepts about dividing a co-created benefit, and develops an indirect approach to performing a Shapley value calculation to implement that allocation.

There are multiple advantages of the token-based approach. 
First, the utility is not committed to compensating consumers more than the total net benefit arising from their collective behavior. In particular, if consumers exhibit a short-term behavior that is highly profitable to them but has a longer-term negative system impact, this will be revealed in reduced token values and diminish the strength of the near-term incentives relative to the long-term benefits.  
Second, the calculation methods ensure that all parties are compensated fairly for their marginal contributions to the total net benefit, i.e., a consumer whose actions increased the total surplus greater will see a larger reward than one whose actions were detrimental. Actions that are both individually and collectively beneficial are rewarded the most, and ones harmful to both are rewarded the least, while those that only benefit one or the other are rewarded somewhere in between.  
Finally, the use of tokens might open up new opportunities. For instance, customers could receive tokens for other activities which are beneficial for the system, such as sharing data or for certain energy efficiency investments.
A discussion of the challenges of working with tokens can be found in \cref{sec:impl_challenges}.

%TESS proposes two additional settlement systems based on tokens, supporting these approaches to settlement in the same manner they were implemented previously. In addition, TESS introduces a token or coin settlement system to permit the use of new digital or crypto-currencies, if desired.
%The first approach is useful in situations where demand and DER flexibility yields a system benefit, such as discharging a battery to reduce bulk power consumption during a wholesale coincident peak hour. This approach uses cooperative game theory concepts about dividing a co-created benefit, and develops an indirect approach to performing a Shapley value calculation to implement that allocation. A certain number of tokens are issued each month by the utility based on the expected value of the total welfare surplus from operations. Customers then participate in the transactive system using these tokens, so their token balances reflect their decisions and actions. At the end of the month the total surplus is calculated and used to assign a value to the tokens. Each participant is then compensated according to how many tokens they hold and their individual bill credits awarded accordingly.
%The advantages of this approach are two-fold. First, the utility is not committed to compensating consumers more than the total net benefit arising from their collective behavior. In particular, if consumers exhibit a short-term behavior that is highly profitable to them but has a longer-term negative system impact, this will be revealed in reduced token values and diminish the strength of the near-term incentives relative to the long-term benefits.  Second, the calculation methods ensure that all parties are compensated fairly for their marginal contributions to the total net benefit, i.e., a consumer whose actions increased the total surplus greater will see a larger reward than one whose actions were detrimental. Actions that are both individually and collectively beneficial are rewarded the most, and ones harmful to both are rewarded the least, while those that only benefit one or the other are rewarded somewhere in between.  
%However, this approach may be deemed too risky by some utilities. In addition, some uncertainty remains regarding whether a token-based settlement mechanism might run afoul of government financial regulations. This concern required the development of a second financially equivalent mechanism, wherein customers bid using the local currency and at the end of the billing cycle, all participants are charged a share of the total cost of operating the system in proportion to their net earnings under the transactive system. The result of this calculation is the same as the one described above. However, the perception of consumers may be significantly different, which is a question for further research. 



