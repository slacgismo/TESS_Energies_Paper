\section{Implementation in an Existing Environment}\label{sec:challenges}

In the following section, we summarize the general setup of the TESS system at HCE (\cref{sec:hce_market_setup}) and describe how we position TESS in the existing tariff system (\cref{sec:position_tariff_system}) as well as our approach to designing device bidding functions (\cref{sec:HCE_design}), followed by the presentation of our financial settlement process (\cref{sec:hce_settlement}). We then describe the challenges associated with such a brownfield deployment (\cref{sec:impl_challenges}) and conclude with a description of planned extensions (\cref{sec:extensions}).

\subsection{General Setup}\label{sec:hce_market_setup}

We specify the technical setup of TESS as described in \cref{sec:technical_setup} to largely implement the standard design of the TS. All participating homes are represented by home hubs that bid on behalf of the customers. In terms of DER, we include photovoltaics, electric vehicle charging, as well as electric storage. Each participating device is further connected to a controller that can provide relevant physical information (such as current solar generation or the state-of-charge of electric storage) to the home hub for the purpose of bidding and implement the subsequent market allocation. Every five minutes, the home hubs send bids to the market operator. In addition, in the roles of the retailer and grid operator, HCE provides information on the supply cost, available import capacity, and unresponsive load. Bids are then cleared in a centralized double auction. 

The market result is characterized by an equilibrium price and the share of the marginal bid (`mode'). The latter equals 1 if the marginal bid (i.e. the bid(s) that determine the market price) can be fully cleared and is between 0 and 1 if that is not the case. For instance, if solar systems are generating more than can be exported, the bid will only partially clear and solar systems must reduce their generation to, for instance, 80\%.

Once, the market is cleared, home hubs compare their devices' bids to the equilibrium price. If the bid price exceeds the equilibrium price (for demand) or is less (for supply), the bid was cleared. 
If price and bid are equal, the bid was marginal and might be only partially cleared, corresponding to the share of the marginal bid.
This information is picked up by the device API and implemented physically, if possible.
%The `mode' equals 1 if the marginal bid (i.e. the bid(s) that determine the market price) can be fully cleared and is between 0 and 1 if that is not the case. For instance, if solar systems are generating more than can be exported, the bid will only partially clear and solar systems must reduce their generation to, for instance, 80\%. The controllers in the HCE system can implement such a reduction. 

All agents are implemented on the cloud and values (measurements, bids, market results) are stored in the TESS database, for validation and billing purposes.

\subsection{Positioning TESS in the Current Tariff System}\label{sec:position_tariff_system}

As HCE is a member-owned electric cooperative, they do not require approval by the relevant utility commission to change and adjust their tariffs. However, the management is responsible to their members and, in general, any substantial tariff changes increase the complexity of the underlying process, including metering, accounting, and billing. 

%Any changes should therefore be well designed and well argued and should have advantages for both HCE and its customers. 
%Therefore, for HCE, participation in TESS was designed as a complement to other rates in HCE's tariff, not a substitute for existing rates. 

In some previous TS designs, participants were subscribed to a distinct transactive tariff (e.g. \citet{Widergren2014}).
We think, however, that a well-designed integration of the TS into the existing tariff system facilitates the deployment of TS in real-world settings. Specifically, we propose to leverage HCE's existing tariff Distribution Flexibility Program (DFP). Under the current design of the DFP, customers grant control over their DERs to HCE and, in return, receive a bill credit. This bill credit is calculated based on the type and number of DERs included into the program as well as their performance. TESS would replace the current mechanism of controlling DERs and determining the applicable bill credit: instead of the current HCE control, DER dispatch would now be price-based, determined through the TS. Also, the applicable bill credit would be determined based on the payments received by the TS. As such, TESS could be seen as a specification of operating the Distribution Flexibility Program, fitting TESS neatly into the existing tariff system and avoid the creation of new tariff categories.
%The ``stacking'' of multiple rates for DER owners changes their outside options relative to a simple greenfield deployment. 
Other tariffs, in particular the fixed retail rate and net metering, remain active, and DER owners can be on multiple rates simultaneously. 

We illustrate the stacking of tariffs with the following example. A customer pays a fixed retail rate on his electricity consumption during the billing period. 
If he also owns a solar panel (potentially procured through DERSA), he can additionally take advantage of net-metering. Under net-metering, the fixed retail rate only applies to his net consumption, i.e. the gross electricity consumption minus the electricity generated by the solar panel. 
Finally, the customer can opt in to participate in DFP. A bill credit would compensate the customer for any inconvenience from load curtailments or foregone payments from solar generation, if curtailed.
%Currently under the DFP, if HCE decided to curtail the solar panel temporarily to cure an issue in the system, this would reduce solar generation and increase the customer's bill. However, thanks to the participation in the DFP, the customer also receives a bill credit that should ideally compensate for the experienced losses. The design of TESS in this brownfield environment must take these complex interactions between rates into account, because without tariff redesign they represent the true alternatives and opportunity costs facing customers.

\subsection{Designing Bidding Functions in an Existing Tariff Structure}\label{sec:HCE_design}

Stacking multiple pre-existing retail rates for DER owners affects how customers participate in the TS. Here we present our design of a TS under the incentive structure imposed by the existing tariff regime as detailed in \cref{sec:HCE_tariff}. 
In previous TS implementations such as \citet{hammerstrom_2008}, the implementation of the TS disregarded the existing tariff system. As the system was fully integrated, customers were not able to opt out but received a side-payment that ensured that customers did not experience any losses.

Two problems arise in the model of a closed system with side payment. 
First, while side payments are possible in an experimental context, they are a major obstacle to scaling up a TS in the real world. 
Second, if outside options were not incorporated into bidding with an open system, the implemented bidding strategies would be sub-optimal for the customer, i.e. they would not be incentive-compatible, and customers would either override or leave the TS. Even more, customers would switch to third parties such as load aggregators which could then arbitrage between the existing tariff and the TS, resulting in a sub-optimal equilibrium.
Designing incentive-compatible bidding function is therefore a crucial requirement to improve consumer welfare and open up the system to other stakeholders such as manufacturers of smart home systems and devices or load aggregators.

To demonstrate the general approach of economic bidding with outside options, we focus on photovoltaics.
Usually, customers with solar panels follow the rationale to use and/or export any electricity that the panel is generating.
Given net metering, any additional kWh generated decreases the customer's bill by the retail rate.\footnote{Under net metering, the marginal value of export and the marginal value of consumption are identical and, under normal conditions, equal the fixed retail rate.}
Given installed capacity, PV generation is exogenous and the actual generation depends on the weather. The customer could, however, disconnect the system or decrease infeed, but he does not have any incentive to do so as that would imply missing out on substantial bill reductions, as described earlier. 
%The payment for solar generation (in terms of bill reductions) under the net metering tariff is granted to TESS customers.
The customer will therefore generally generate the maximum electricity possible, given weather conditions and installed capacity.

The customer can, however, provide flexibility to the system by decreasing solar generation or increasing net load at the rate of current production. 
The combination of the PV technology behavior and the NEM rate describes the baseline against which the customer will evaluate any changes he makes to his behavior, and he will take that baseline into account when making a bid.
Importantly, curtailment would result in an increase of his bill by the fixed retail rate for each kWh not generated because of the curtailment. The individual participation constraint requires him to receive at least the retail rate - he will only alter his behavior if that will make him better off. Therefore, the customer would place a bid in the TS to reduce supply (or increase net load) by the current generation of the solar panel at a minimum price of the retail rate.

In the case of solar generation, the amount of flexibility would correspond to the current solar infeed $q_{PV}$ because, at best, the customer could fully shut down his solar generation. The price bid for this deviation would be at least the fixed retail rate $RR$ as this is the bill reduction the customer would be losing in case of the shutdown of his PV system. Vice versa, he is not able to offer any negative baseline reductions, i.e. increasing solar generation. \cref{fig:lem_adjustment_normal} illustrates this baseline effect.

Under normal conditions, of course, HCE would not be willing to pay a customer the retail rate to reduce generation. Instead, it would be willing to sell additional generation at its long-term marginal supply cost $MC$ (and even take a price less than the fixed retail rate). On the other side, it would be only willing to reduce supply from the baseline scenario if paid the foregone profit, i.e. the difference between the retail rate and the marginal cost. The bids provided by HCE and consumers are illustrated in \cref{fig:lem_adjustment_normal}. The curves do not intersect and the market allocation is empty.

A situation is possible, however, during which HCE would want to decrease generation within its distribution grid or, equivalently, increase net load. Such a situation could arise when there is an abundance of renewable energy in the grid. Then, the entity operating the overlying grid could be willing to pay a critical peak rate $CPR$ to HCE for net load increases, or HCE would need to curtail renewable energy within its system and pay them a compensation of $CPR$. In that case, the HCE bid changes and it would be willing to sell at a negative price, $MC - CPR$, i.e. pay consumers to consume more. Further decreasing supplies would be even more costly for the marginal unit, as HCE would miss out on sales at the fixed retail rate. 
The bids provided by HCE and consumers are illustrated in \cref{fig:lem_adjustment}. Now, supply and demand adjustments intersect at a quantity of $q_{PV}$ and the clearing price $p$ would be determined between $p \in [MC - CPR, -RR]$. Because the price is negative, the cashflow reverses and buyers get paid, while the seller pays.

\begin{figure}[h]\label{fig:baseline}
  \begin{subfigure}[t]{0.485\linewidth}
    \centering\includegraphics[height=3.2cm]{Market_power_adjustment_normal_2.png}
    \caption{Market under normal supply conditions}
    \label{fig:lem_adjustment_normal}
  \end{subfigure}\hspace{0.5cm}
  \begin{subfigure}[t]{.485\linewidth}
    \centering\includegraphics[height=3.2cm]{Market_power_adjustment_2.png}
    \caption{Market under requirement to increase net load} 
    \label{fig:lem_adjustment}
  \end{subfigure}
\caption{Market for power adjustment (compared to baseline)}
\end{figure}

Importantly, given the outside options of customers (in particular net metering), our design of the TS does not trade energy or power, but rather the deviation from this baseline generation or consumption. Thus the clearing price on the market reflects the price of \textbf{deviating} from the baseline. As a result, bids must reflect the quantity by which customers would be willing to deviate from the baseline and at what price. 

\cref{fig:pv_profile} illustrates the resulting implementation of the market result for a single customer with a solar system. For demonstration purposes, solar generation is approximated as a continuous normal distribution function. Generation starts in the early morning hours, reaches its maximum around noon, and terminates after sunset. The home hub, on behalf of the controller, continuously provides bids for increasing net load/reducing generation at the respective current generation $q_t$, according to the described bidding strategy. Now assume that an adverse event occurs at 1pm, the market is cleared, and the solar system is shut down. In that case, solar generation is reduced to zero. In the subsequent market interval, no current (counterfactual) generation is available. For the implementation at HCE that relies on deviation from a baseline, we therefore approximate the available load reduction (i.e. to keep the solar system curtailed) using the most recent measurement. Therefore, the reduction for which the customer is eventually paid at settlement, corresponds to the red straight line of counterfactual generation during the adverse event. Once the situation dissolves, the market does not clear anymore and the solar system goes back online.

\begin{figure}
\centering
\includegraphics[scale=0.8]{TESS_PV_reduction.png}
\caption{PV curtailment}
\label{fig:pv_profile}
\end{figure}

TESS will integrate other DERs, specifically electric vehicle chargers and electric storage, at a later stage of the field deployment. Other device bidding functions will reflect the technical capabilities of each device and the opportunity costs and outside options of their owners. The baseline bidding strategies of PV panels as well as other DER come with significant challenges which will be discussed in \cref{sec:impl_challenges}.

\subsection{Settlement in TESS}\label{sec:hce_settlement}

In the last section, we described the bidding and market clearing in monetary terms, at given marginal costs and value. An additional problem in real-world systems, however, is that often actual costs/value only become clear in real-time or even  \textit{ex post}. One example is the cost of consuming or the value of producing during a coincident peak event. In the case of HCE, coincident  peak events happen at the time of the maximum net load in the Xcel system within a month. During the coincident peak, marginal consumption is significantly more expensive than during the rest of the month. However, it is unclear when it will actually happen within a month. For instance, overall load could be forecasted to be very high during a day of the first week and the marginal supply costs would be supposedly very high, i.e. $MC_{coin} >> MC$. However, it might turn out that, later during the month, aggregate demand would be even higher. In that case, \textit{ex post}, the supply adjustment cost bid by HCE in the TS would have been wrong and any load adjustments would have been less valuable as previously thought. As a results, the bill credit required to be granted by HCE to responding customers would exceed the actual value they provided to the system.

Therefore, TESS will support tokens as a substitute for USD to bid in the TS. For a first implementation of our TS, we assume that customers have a prior of 1 token corresponding to the value of 1 USD. Therefore, any bids they make in USD could be translated into token bids on a 1:1 rate. 
At the end of the billing period, however, the market operator (here: HCE) would be able to calculate the total turnover of tokens during the month and determine the actual token value by comparing the turn-over to the savings achieved thanks to the operations of the TS. Multiplying the tokens collected by a customer with the final token value would give the final bill credit.

For the monthly bill of the customer, all tariffs need to be included again (i.e. fixed retail rate as well as net-metering). The general electricity bill will be based on the read-out from the net meter, i.e. the net consumption during the billing period, multiplied by the fixed retail rate. In addition, TESS customers will be granted a bill credit, as foreseen in the DFP specification. Instead of a fixed amount per participating DER, the bill credit will depend on the turn-over of tokens the device realized within TESS, multiplied by the value of a token. If, for instance, the solar panel of a customer disconnected due to an adverse event, the customer receives the product of the increase in net demand/reduction in generation as compared to the base line, the token price paid by HCE for this deviation, and the token value in USD. In general, customers would not be willing to participate for a price below the fixed retail rate because of the net metering outside option, so their losses due to decreased solar generation will at least be off-set by the aggregate bill credit.

\subsection{Implementation Challenges}\label{sec:impl_challenges}

The implementation of TS in brownfield deployments comes with several challenges.

First, baseline approaches (i.e. providing demand flexibility as compared to a usual/expected load baseline) usually suffer from gaming. In other words, customers could influence their baseline calculation to be able to report a bigger deviation from it and receive higher payments. In the case of PV, this risk is limited as HCE has access to the rated power of the system bidding as well as the PV measurements prior to the event. The latter can serve as a myopic forecast for the limited duration of a possible event requiring to increase net load. Moreover, HCE has access to data of solar panels that are not participating in TESS which provide a control baseline. Given these possibilities to validate customers' bids, we believe it is reasonable that customers bid their possible deviation to the best of their knowledge and specify the home hub to bid the latest measurement of PV generation as a possible deviation during the time when the system has not been curtailed. 
The problem becomes, however, more severe for other devices such as electric vehicles. When other devices should be included in such a brownfield implementation, TS designed need to specify how a baseline can be determined and how, for instance, information from such devices' bidding behavior can be used to verify it.

Second, even if a baseline can be established, the baseline approach changes the distribution of surplus between the utility as well as customers with different appliances. In the case of solar, customers will still enjoy the benefits of net-metering, even at time, when the value of generation is far below the level of the fixed retail rate. Such effects can lead to inefficient investment in the long-term, as compared to a greenfield deployment with marginal value/cost bidding.

Third, the introduction of tokens opens up new opportunities, but also challenges. With regard to opportunities, tokens can be used to further reimburse customers for other beneficial actions like the provision of data and energy efficiency investments. In theory, this could be accomplished in USD, but other programs like Ohm Connect or recent insight in gamification have shown that such tokens provide greater operational flexibility and might be better able to incentivize system-friendly behavior.
On the other hand, the introduction of tokens introduces another layer of uncertainty with regard to token value. Therefore, bidding should actually be based on the expected token value rather than on a fixed ``myopic'' assumption with respect to the value of a token.
In addition, some uncertainty remains regarding whether a token-based settlement mechanism might run afoul of government financial regulations.

Finally, a TS alters the cash-flow of a utility. In non-TS, retail rates have been calculated to cover expected cost projections (as well as a risk premium), including increased costs during adverse events. The introduction of a TS provides the utility with more control capabilities and enables cost savings. The additional financial means -- i.e. the difference between the income from customer bills and the supply and operational costs -- should be able to cover the relevant bill credits. 
Established accounting model must be adjusted during a period of introducing a TS.

\subsection{Planned Extensions}\label{sec:extensions}

The current implementation of TESS at Basalt Vista on HCE territory represents a minimum viable product with photovoltaics and a double-sided centralized auction. In the following, we describe the planned extensions which, we think, will provide the most immediate value.

First, we are planning to incorporate electric vehicle charging as well as electric storage in Basalt Vista. 
Second, we plan to re-design bidding on behalf of HCE to reflect their planned participation in the Western balancing market. This will mean that they do not face distinct supply costs, depending on if the system operated under `normal' or `peak' conditions, but for which costs can change in real-time.
Third, we are planning to design an incentive-compatible settlement mechanism that ensures implementation of market results, even in an open TS.
Finally, we plan to extend TESS to support up to three distinct price discovery mechanisms simultaneously, i.e., energy prices (\$/MWh), storage prices (\$/MWh$^2$), and ramping prices (\$/MW).


