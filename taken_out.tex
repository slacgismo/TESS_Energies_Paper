We are currently implementing and operating a TS in Basalt Vista, a community managed by the cooperative Holy Cross Energy (HCE) in Colorado. At the moment, HCE procures energy at the wholesale level from Xcel Energy, at a fixed rate. Under the terms of that contract, significant cost savings arise for HCE from reducing load during the monthly coincident peak of the system. Customers are subject to a fixed retail tariff and net metering, if they own solar. In the future, Colorado is set to enter the Western energy imbalance markets, which will open up new cost-reducing opportunities for active load management. Furthermore, the increased deployment of renewable energies as well as stability-threatening events like wildfires will require more load management capabilities. These changes will consequently change the value attached to resource flexibility, and to network architectures that can facilitate flexibility and enable HCE to discover the value of flexibility and how it varies over time and location.

In this context, TS can provide valuable means for decentralized and flexible load coordination. Here we describe the cooperative and the distribution system of interest (\cref{sec:HCE_description}), the requirements and limitations for the design of our TS (\cref{sec:HCE_limitations}), and the design we propose (\cref{sec:HCE_design}).

\textcolor{blue}{opportunity cost, outside option, PV bidding function set at its opportunity cost MC=0 means no strategic bidding but also no preference elicitation, some existing programs have baselines, representing preferences, continuing to apply mechanism design ideas as we adapt the design to their situation}

Implementing TESS in a utility setting requires modifications to accommodate the existing tariff and rate structure. In a TS implementation without an existing institutional framework, simply implementing the technology, market design, and bidding functions could be feasible. Even in a simple setting with a single fixed rate tariff, TESS would provide a contractual alternative, so the customer's evaluation of their opportunity costs would be straightforward. The TESS market design would also be more straightforward -- devices would submit their demand bids and supply bids, the utility would submit its supply bid, and market clearing and coordination would proceed as described in \ref{sec:standard_design}.

In HCE's case, customers have a range of rate options if they own DERs, and some of those rates can be combined. Customers who own or are considering investing in DERs remain on the base residential rate, then add on NEM and DFP, and DERSA if they purchase their DERs through HCE.

Doesn't change the essentials of the design, but it does change the definition of what's being traded in the market -- e.g., have to trade differences compared to a baseline

Importantly, our specification addresses challenges of implementing TS in existing electric retail systems, for instance, the design of bidding strategies when a (non-transactive) tariff system is already in place.

Transactive energy synthesizes the technologies and control theory from engineering with the market design and institutional theory from economics into a cyber-physical-social system platform. Digital technologies create the possibility of energy devices changing their settings autonomously in response to prices, and control theory provides a framework for modeling, testing, and understanding the individual and system effects of such capabilities.

TESS provides automated retail market clearing mechanisms and price-based dispatch of behind-the-meter distributed energy resources. 



Digitally-enabled devices; market-clearing price serves as signal for autonomous device control and price-based dispatch; devices algorithmically change settings

VALUE OF TESS FOR UTILITIES SEEING THREATS TO THEIR EXISTING BUSINESS MODEL -- REVENUE, COST, RESILIENCE. RELEVANT TO IOU, COOP, MUNI. WAY TO MITIGATE IMPACTS OF EMERGING RISKS BY ENABLING MORE AND DEEPER FLEXIBILITY

devices can act autonomously on the preferences and objectives of the people who own or operate them

Despite market design challenges, the combination of environmental policy and falling production costs has driven a proliferation of distributed energy resources in electric power systems. Transactive energy is emerging as a fundamentally new market-based approach to coordinating electric energy delivery in systems with very high levels of DERs. Transactive energy combines technology and engineering with market design and economic principles. Autonomous device response to informative, emergent prices is the hallmark of transactive energy.


In particular, the TESS design embodies the economic concept that the transactive process is one of \emph{price discovery} through the mutual dynamic interaction of demand and supply, of value and cost. In that process, agents evaluate their opportunity costs, and in TESS those evaluations will be embodied in the algorithmic bidding functions of different devices. An important aspect of individual opportunity costs is the range of options facing a person making a choice. In the modifications of TESS that we performed to adapt to a brownfield setting, the presence of existing outside options was a crucial factor to incorporate into the modified market design and bidding functions.


\citet{sajjadi_transactive_2016} base their bid on the average cost of a kWh generated, including total cost of installation. We do not follow their approach as this number does not capture the effects on individual incentives of the marginal benefit of participating in the TS, nor does it capture additional existing tariff conditions as outside options. In a European context, \citet{ableitner_user_2020} implemented a local electricity market. While owners of solar panels could sell at a guaranteed feed-in-tariff, they find that some would be willing to sell at lower costs. \citet{mengelkamp_decentralizing_2018} automate customers to bid marginal supply costs between the feed-in-tariff and the retail rate.

In the context of fixed feed-in tariffs, comparable considerations have been made by \citet{mengelkamp_decentralizing_2018} who automate customers to bid marginal supply costs between the feed-in-tariff and the retail rate. 

The goal of the TESS project is to design, develop, test, and validate retail-level Transactive Energy systems that are dominated by behind-the-meter renewable energy resources and energy storage resources. Our definition differs from the cost-based ``prices to devices'' concept by including demand-side and DER bidding based on willingness to pay -- eliciting ``prices from devices'' to reflect demand-side value more directly in the market-clearing prices used for dispatch.

The information systems support a wider range of capabilities such that any  bidding function is permitted, instead of the ``approved'' functions used in the Olympic and subsequent transactive system demonstrations. Accommodating heterogeneous bidding functions reflects the subjective nature of preferences and opportunity costs and enables a more decentralized platform that gives utilities, device manufacturers, and consumers more flexibility and autonomy than previously possible in transactive systems.

%TESS proposes two additional settlement systems based on tokens, supporting these approaches to settlement in the same manner they were implemented previously. In addition, TESS introduces a token or coin settlement system to permit the use of new digital or crypto-currencies, if desired.
%The first approach is useful in situations where demand and DER flexibility yields a system benefit, such as discharging a battery to reduce bulk power consumption during a wholesale coincident peak hour. This approach uses cooperative game theory concepts about dividing a co-created benefit, and develops an indirect approach to performing a Shapley value calculation to implement that allocation. A certain number of tokens are issued each month by the utility based on the expected value of the total welfare surplus from operations. Customers then participate in the transactive system using these tokens, so their token balances reflect their decisions and actions. At the end of the month the total surplus is calculated and used to assign a value to the tokens. Each participant is then compensated according to how many tokens they hold and their individual bill credits awarded accordingly.
%The advantages of this approach are two-fold. First, the utility is not committed to compensating consumers more than the total net benefit arising from their collective behavior. In particular, if consumers exhibit a short-term behavior that is highly profitable to them but has a longer-term negative system impact, this will be revealed in reduced token values and diminish the strength of the near-term incentives relative to the long-term benefits.  Second, the calculation methods ensure that all parties are compensated fairly for their marginal contributions to the total net benefit, i.e., a consumer whose actions increased the total surplus greater will see a larger reward than one whose actions were detrimental. Actions that are both individually and collectively beneficial are rewarded the most, and ones harmful to both are rewarded the least, while those that only benefit one or the other are rewarded somewhere in between.  
%However, this approach may be deemed too risky by some utilities. In addition, some uncertainty remains regarding whether a token-based settlement mechanism might run afoul of government financial regulations. This concern required the development of a second financially equivalent mechanism, wherein customers bid using the local currency and at the end of the billing cycle, all participants are charged a share of the total cost of operating the system in proportion to their net earnings under the transactive system. The result of this calculation is the same as the one described above. However, the perception of consumers may be significantly different, which is a question for further research. 

%The physical scaling of a transactive system can also be a difficult design and implementation problem for utilities, although usually it is not a problem. For example, in the Columbus demonstration project \citep{Widergren2014} the standard transactive system was deployed on 4 feeders without significant changes to the implementation.  However, the Pacific Northwest project \citep{hammerstrom2015pacific} encountered a number of important market design and operation issues that emerged due the need to coordinate the dispatch of retail resources across 11 utilities in a region without restructured electricity markets.
%
%TESS continues to focus on retail-level pricing mechanisms, but does not focus exclusively on using energy prices (\$/MWh) to manage distribution system constraints.  As mentioned in \ref{sec:teecon}, TESS is designed to permit discovery of other prices, such as storage prices (\$/MW$^2$) or ramping prices (\$/MW) that can be used to provide signals that can incentivize changes to the state of charge objective of a battery or the power setting of an inverter, respectively.  However, these prices signals have yet to be studied in detail in simulations or demonstrated in the field, and will be subject of future projects.

%The `mode' equals 1 if the marginal bid (i.e. the bid(s) that determine the market price) can be fully cleared and is between 0 and 1 if that is not the case. For instance, if solar systems are generating more than can be exported, the bid will only partially clear and solar systems must reduce their generation to, for instance, 80\%. The controllers in the HCE system can implement such a reduction. 

%Currently under the DFP, if HCE decided to curtail the solar panel temporarily to cure an issue in the system, this would reduce solar generation and increase the customer's bill. However, thanks to the participation in the DFP, the customer also receives a bill credit that should ideally compensate for the experienced losses. The design of TESS in this brownfield environment must take these complex interactions between rates into account, because without tariff redesign they represent the true alternatives and opportunity costs facing customers.

